\documentclass[useAMS,usenatbib]{mn2e}
\usepackage{graphicx}
\usepackage{amssymb}
\usepackage{natbib}
\bibliographystyle{mn2e}
\pdfminorversion=5   % recommended by MNRAS web page


%%% Journal abbreviations.
\def\apj{ApJ}                 % Astrophysical Journal
\def\apjl{ApJL}               % Astrophysical Journal, Letters
\def\apjs{ApJS}               % Astrophysical Journal, Supplement
\def\mnras{MNRAS}             % Monthly Notices of the RAS
\def\aap{A\&A}                % Astronomy and Astrophysics
\def\aaps{A\&AS}              % Astronomy and Astrophysics, Supplement
\def\aj{AJ}                   % Astronomical Journal
\def\physrep{Phys.~Rep.}      % Physics Reports
\def\nat{Nature}              % Nature
\def\araa{ARA\&A}             % Annual Review of Astronomy and Astrophysics
\def\planss{planss}           % Planetary and Space Science
\def\ssr{SSR}                 % Space Science Reviews
\def\sovast{Sov.~Astron.}     % Soviet Astronomy
\def\canjphys{Can. J. Phys.}  % Canadian Journal of physics
\def\nar{New~Astron.}

 
\newcommand{\msun}{{M$_\odot$}}

\topmargin -1cm


\title{Cooling Clouds by Varying Metallicities: Origin of Globular Cluster Bimodality}

\author[R. Fernandez et al.]{Ricardo Fernandez$^{1}$ and Greg L. Bryan$^{1}$\\
$^{1}$Department of Astronomy, Columbia University, 550 West 120th Street, New York, NY 10027, USA}

\begin{document}

\date{}

%\pagerange{\pageref{firstpage}--\pageref{lastpage}} \pubyear{2013}

\maketitle

%\label{firstpage}

\begin{abstract}
Globular Clusters
\end{abstract}

\begin{keywords}
globular clusters - methods:numerical
\end{keywords}

%% ----------------------------------------------------------------
%
\section{Introduction}

%% ----------------------------------------------------------------
%
\section{Basic Idea}
\label{sec:basic}

%% ----------------------------------------------------------------
%
\section{Numerical Models}
\label{sec:numerical}
\subsection{Numerical Method}
This simulations were performed with the publicly available Eulerian three-dimensional
hydrodynamical adaptive mesh refinement Enzo code \citep{Bryan2013}. The domain
box size of the simulation was 150 pc with a top level root grid resolution of $128^3$.
Cell refinement was dictated by baryon mass and Jeans length with a maximum refinement
level of 3. Our simulations included self gravity and radiative cooling using the
Grackle library; details described in \cite{Bryan2013}. The metal heating and
cooling rates are provided from \cite{Haardt2012}.

\subsection{Initial Conditions}
Our initial conditions consisted of a cloud in pressure equilibrium with an
ambient density and temperature background. The internal structure of the
cloud is modeled by a Bonner-Ebert sphere \cite{Bonnor1956}; a self-gravitating
isothermal gas sphere in hydrostatic equilibrium embedded in a pressurized  
medium. To fully describe a Bonner-Ebert sphere, a mass $M_{BE}$, temperature
$T_{BE}$, and an external pressure $P_{ext}$ must be chosen. Following our
assumptions outlined in Section~\ref{sec:basic}, we choosed
$M_{BE}=10^6$ \msun, $T_{BE}=6000$ K, and $P_{ext}=1.8\times10^5\times k_B$
($k_B$ : Boltzmann constant). This corresponds to a cloud on the cusp where
heating and cooling balance. 

In addition, we add turbulence to the cloud following a power spectrum of
$v_k^2 \propto k^{-4}$ for the velocity field.

%% ----------------------------------------------------------------
% 
\section{Results}
\label{sec:results}
\subsection{No Heating Runs}
\subsection{Heating Runs}

%% ----------------------------------------------------------------
% 
\section{Discussion}
\label{sec:discussion}
\subsection{Analytic Model}
\subsection{Implications}
\subsection{Caveats}

%% ----------------------------------------------------------------
% 
\section{Summary}

%% ----------------------------------------------------------------
% 
\section*{Acknowledgments}
\bibliography{mn-jour,gc_paper}
%
%\label{lastpage}
\end{document}
